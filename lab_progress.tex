\documentclass[10pt]{beamer}

\usetheme{m}

\usepackage{booktabs}
\usepackage[scale=2]{ccicons}

\usepackage{pgfplots}
\usepgfplotslibrary{dateplot}

\usepackage{graphicx}
\graphicspath{{img/}}

% strike through
\usepackage[normalem]{ulem}

% bitbucket icon
\usepackage{fontspec}
% fontawesome path
% \defaultfontfeatures{
%     Path = /usr/local/texlive/2015basic/texmf-dist/fonts/opentype/public/fontawesome/ }
\usepackage{fontawesome}

% color
\usepackage{xcolor}

\newcommand{\vfillall}{\vskip0pt plus 1filll}

\title{Efficient Gyro-Roller Based Rehabilitation Program For Stroke Patients}
%\subtitle{A modern beamer theme}
\date{}
\author{Tulakan Ruangrong}
\institute{AIMLAB - Biomedical Engineering - Mahidol University}
% \titlegraphic{\hfill\includegraphics[height=1.5cm]{logo/logo}}

\begin{document}

\maketitle

\begin{frame}
  \frametitle{Table of Contents}
  \setbeamertemplate{section in toc}[sections numbered]
  \tableofcontents[hideallsubsections]
\end{frame}

\section{Introduction}

\begin{frame}
  \frametitle{Stroke}
  	\hfill
	\newline
	\textbf{STROKE}
	\begin{itemize}
		\item Around \alert{\emph{20,000 deaths}} in Thailand every year.
		\item Major cause of paralytic.
	\end{itemize}
	\begin{figure}[h]
	\centering
	\includegraphics[width=0.7\textwidth]{stat}
	\caption{Global deaths from Cardiovascular disease}
	\end{figure}
\end{frame}

\begin{frame}{Rehabilitation}
	\textbf{REHABILITATION}
	\begin{itemize}
		\item \alert<1>{Neural plasticity}
		\item \alert<2>{Most of commercial devices are very expensive}
		\item \alert<3>{Strict and repetitive process}
		\item \alert<4>{Easy to be motiveless and bored}
	\end{itemize}
	\hfill
	\hfill
	\begin{center}
		\alert<5>{\uppercase{\Large{\emph{\textbf{Combination between virtual reality and rehabilitation techniques}}}}}
	\end{center}
\end{frame}

\begin{frame}{Gyro-Roller}
	\begin{columns}[c]
		\column{0.65\textwidth}
		\begin{figure}[h]
			\includegraphics[width=\textwidth]{gyrosys}
			\caption{Gyro-Roller System}
		\end{figure}
		\column{0.35\textwidth}
		\begin{figure}[h]
			\includegraphics[width=\textwidth]{patient1}
			\newline
			\break
			\includegraphics[width=\textwidth]{patient2}
			\caption{With patients}
		\end{figure}
	\end{columns}
\end{frame}

\begin{frame}{Gyro-Roller}
	\textbf{Difference between 2nd and 3rd version}
	\break
	\begin{columns}[c]
		\column{0.5\textwidth}
		\begin{figure}
			\includegraphics[width=\textwidth]{wheel_ver2}
			\caption{Version 2 wheel}
		\end{figure}
		\column{0.5\textwidth}
		\begin{figure}
			\includegraphics[width=\textwidth]{wheel_ver3}
			\caption{Version 3 wheel}
		\end{figure}
	\end{columns}
\end{frame}

\begin{frame}{Thesis Objectives}
	\hfill
	\begin{itemize}
		\large{
		\item Game Design
		\item Virtual Reality based Gyro-Roller system
		\item Clinical Trial
		}
	\end{itemize}
	\begin{figure}[h]
		\includegraphics[width=0.4\textwidth]{elder}
		\,\,\,\,
		\includegraphics[width=0.3\textwidth]{child}
	\end{figure}
\end{frame}

\begin{frame}{Thesis Scopes}
	\begin{itemize}
		\large{\alert<1>{\item Develop 3 different games with active \& passive modes including several levels and log file.}
		\alert<2>{\item Find out how effective of the Gyro-Roller version 3 over version 2.}
		\alert<3>{\item Collect the data of 20 subjects for at least 2 months.}}
	\end{itemize}
\end{frame}

\section{Previous Works}
% new command used - creating strikethrough function
% \newcommand{\sttr}[1]{\rlap{\rule[0.5ex]{6em}{0.1ex}}\text{#1}}


\begin{frame}{Problem Solved}
	\large \textbf{Mechanic}
	\normalsize
	\begin{itemize}
		\item \sout{Fix pulley belt tension}
		\item \sout{Fix handle bar alignment}
		\item \sout{Wiring servomotor -> tuning goal position}
	\end{itemize}
	
	\large \textbf{Software}
	\normalsize
	\begin{itemize}
		\item \sout{Write new Arduino sketch to control DC motor}
	\end{itemize}
\end{frame}

\begin{frame}{Game Development}
	\vfillall
	\large \textbf{Game pages – integrated}
	\begin{itemize}
		\item Login
		\item Registration
		\item Game Selector
		\item Calibration – with motor connected
		\item EMG collection game 
		\item Space shooting game – \alert{being integrated}
	\end{itemize}
	\vfillall
	\centering \footnotesize \# Project source is hosted privately on \,{\color{blue!52!cyan}\faBitbucket \,Bitbucket.org}
	\smallskip
\end{frame}
%--- Next Frame ---%

\section{Present Works}
	\begin{frame}{All}
		\large \textbf{Literature Review}
		\normalsize
		\begin{itemize}
			\item Cognitive rehabilitation
		\end{itemize}
		\large \textbf{Game Development}
		\normalsize
		\begin{itemize}
			\item Add mode to control mass movement
			\item Integrate developed modules
			\item Create cognitive based games
		\end{itemize}
		\large \textbf{EMG Analysis}
		\normalsize
		\begin{itemize}
			\item Figure difference between mass to the left-right
			\item Apply information to the game
		\end{itemize}
	\end{frame}
%--- Next Frame ---%
	
	
	
	









\begin{frame}[fragile]
  \frametitle{Typography}
      \begin{verbatim}The theme provides sensible defaults to
\emph{emphasize} text, \alert{accent} parts
or show \textbf{bold} results.\end{verbatim}

  \begin{center}becomes\end{center}

  The theme provides sensible defaults to \emph{emphasize} text,
  \alert{accent} parts or show \textbf{bold} results.
\end{frame}
\begin{frame}{Lists}
  \begin{columns}[T,onlytextwidth]
    \column{0.33\textwidth}
      Items
      \begin{itemize}
        \item Milk \item Eggs \item Potatos
      \end{itemize}

    \column{0.33\textwidth}
      Enumerations
      \begin{enumerate}
        \item First, \item Second and \item Last.
      \end{enumerate}

    \column{0.33\textwidth}
      Descriptions
      \begin{description}
        \item[PowerPoint] Meeh. \item[Beamer] Yeeeha.
      \end{description}
  \end{columns}
\end{frame}
\begin{frame}{Animation}
  \begin{itemize}[<+- | alert@+>]
    \item \alert<4>{This is\only<4>{ really} important}
    \item Now this
    \item And now this
  \end{itemize}
\end{frame}
\begin{frame}{Figures}
  \begin{figure}
    \newcounter{density}
    \setcounter{density}{20}
    \begin{tikzpicture}
      \def\couleur{alerted text.fg}
      \path[coordinate] (0,0)  coordinate(A)
                  ++( 90:5cm) coordinate(B)
                  ++(0:5cm) coordinate(C)
                  ++(-90:5cm) coordinate(D);
      \draw[fill=\couleur!\thedensity] (A) -- (B) -- (C) --(D) -- cycle;
      \foreach \x in {1,...,40}{%
          \pgfmathsetcounter{density}{\thedensity+20}
          \setcounter{density}{\thedensity}
          \path[coordinate] coordinate(X) at (A){};
          \path[coordinate] (A) -- (B) coordinate[pos=.10](A)
                              -- (C) coordinate[pos=.10](B)
                              -- (D) coordinate[pos=.10](C)
                              -- (X) coordinate[pos=.10](D);
          \draw[fill=\couleur!\thedensity] (A)--(B)--(C)-- (D) -- cycle;
      }
    \end{tikzpicture}
    \caption{Rotated square from
    \href{http://www.texample.net/tikz/examples/rotated-polygons/}{texample.net}.}
  \end{figure}
\end{frame}
\begin{frame}{Tables}
  \begin{table}
    \caption{Largest cities in the world (source: Wikipedia)}
    \begin{tabular}{lr}
      \toprule
      City & Population\\
      \midrule
      Mexico City & 20,116,842\\
      Shanghai & 19,210,000\\
      Peking & 15,796,450\\
      Istanbul & 14,160,467\\
      \bottomrule
    \end{tabular}
  \end{table}
\end{frame}
\begin{frame}{Blocks}
  Three different block environments are pre-defined and may be styled with an
  optional background color.

  \begin{columns}[T,onlytextwidth]
    \column{0.5\textwidth}
      \begin{block}{Default}
        Block content.
      \end{block}

      \begin{alertblock}{Alert}
        Block content.
      \end{alertblock}

      \begin{exampleblock}{Example}
        Block content.
      \end{exampleblock}

    \column{0.5\textwidth}

      \metroset{block=fill}

      \begin{block}{Default}
        Block content.
      \end{block}

      \begin{alertblock}{Alert}
        Block content.
      \end{alertblock}

      \begin{exampleblock}{Example}
        Block content.
      \end{exampleblock}

  \end{columns}
\end{frame}
\begin{frame}{Math}
  \begin{equation*}
    e = \lim_{n\to \infty} \left(1 + \frac{1}{(n+1)^n}\right)^n
  \end{equation*}
\end{frame}
\begin{frame}{Line plots}
  \begin{figure}
    \begin{tikzpicture}
      \begin{axis}[
        mlineplot,
        width=0.9\textwidth,
        height=6cm,
      ]

        \addplot {sin(deg(x))};
        \addplot+[samples=100] {sin(deg(2*x))};

      \end{axis}
    \end{tikzpicture}
  \end{figure}
\end{frame}
\begin{frame}{Bar charts}
  \begin{figure}
    \begin{tikzpicture}
      \begin{axis}[
        mbarplot,
        xlabel={Foo},
        ylabel={Bar},
        width=0.9\textwidth,
        height=6cm,
      ]

      \addplot plot coordinates {(1, 20) (2, 25) (3, 22.4) (4, 12.4)};
      \addplot plot coordinates {(1, 18) (2, 24) (3, 23.5) (4, 13.2)};
      \addplot plot coordinates {(1, 10) (2, 19) (3, 25) (4, 15.2)};

      \legend{lorem, ipsum, dolor}

      \end{axis}
    \end{tikzpicture}
  \end{figure}
\end{frame}
\begin{frame}{Quotes}
  \begin{quote}
    Veni, Vidi, Vici
  \end{quote}
\end{frame}

\begin{frame}{References}
  % Some references to showcase [allowframebreaks] \cite{knuth92,ConcreteMath,Simpson,Er01,greenwade93}
\end{frame}

\section{Conclusion}

\begin{frame}{Summary}

  Get the source of this theme and the demo presentation from

  \begin{center}\url{github.com/matze/mtheme}\end{center}

  The theme \emph{itself} is licensed under a
  \href{http://creativecommons.org/licenses/by-sa/4.0/}{Creative Commons
  Attribution-ShareAlike 4.0 International License}.

  \begin{center}\ccbysa\end{center}

\end{frame}

\plain{Questions?}

\begin{frame}[allowframebreaks]

  \frametitle{References}

  \bibliography{demo}
  \bibliographystyle{abbrv}

\end{frame}

\end{document}
